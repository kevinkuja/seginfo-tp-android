\section{Discusi\'on}

Para que funcione el malware tenemos que agregar ciertas variables al Manifest: el registro del service \code{HijackerService}, y de \code{HijackerServiceLauncherAtBoot} indicando que recibir\'a la acci\'on de \code{BOOT\_COMPLETED}, as\'i como los permisos que tendr\'a la aplicaci\'on: 

\begin{enumerate}[nolistsep]
 \item \code{READ\_CONTACTS} para leer los contactos.
 \item \code{ACCESS\_FINE\_LOCATION} para acceder al gps.
 \item \code{ACCESS\_COARSE\_LOCATION} para acceder a la ubicaci\'on a trav\'es del proveedor de internet.
 \item \code{INTERNET} para poder usar internet.
 \item \code{ACCESS\_NETWORK\_STATE} para leer el estado de la red.
 \item \code{RECEIVE\_BOOT\_COMPLETED} para poder lanzar el malware al iniciar el celular. 
\end{enumerate}

Como decidimos hacer el malware en una sola aplicaci\'on estos son los m\'inimos permisos que se requieren para funcionar y para el juego 2048 se podr\'ia argumentar que son exagerados y un usuario podr\'ia sospechar de instalarlo. Pero como el malware no depende de \'el para funcionar, se podr\'ia poner en cualquier aplicaci\'on que los usara realmente ya que son permisos comunes y b\'asicos ciertas apps, y as\'i el usuario no sabr\'ia que se est\'an aprovechando de estos permisos para robarle informaci\'on. 

Hay que tener en cuenta que para que el malware funcione este debe estar conectado a internet para poder mandar la informaci\'on al server, y el GPS debe estar prendido para obtener una ubicaci\'on m\'as precisa, aunque si no lo est\'a utilizar\'a la ubicaci\'on del proveedor de internet. 

Es parametrizable la frecuencia de robo de la informaci\'on modificando la variable \code{FREQ\_HIJACKER} de la clase \code{HijackerServiceLauncher} y la direcci\'on del server a donde enviar los datos. Esto hay que setearlo antes de compilar la aplicaci\'on. 

De los contactos decidimos robar solamente el nombre y los n\'umeros telef\'onicos por comodidad, pero se podr\'ia agregar cualquier otra informaci\'on que se quisiera. 

A lo largo del desarrollo del malware no tuvimos muchas dificultades, simplemente el testeo de la aplicaci\'on en un emulador que tra\'ia ciertas limitaciones como puede ser que no reconozca el GPS y haya que enviarle la información por otro lado, y el testeo de que se inicie apenas se prenda el telefono. A pesar de eso, en cuestiones de visualización, l\'ogica y acceso a la información a robar no tuvimos problema, lo que nos concientiza sobre lo f\'acil que es modificar una aplicaci\'on y manipular al usuario para hacer fraude sin que este se diera cuenta. 

\subsection{Mejoras}

\begin{itemize}
\item Configuración de la IP del server. 
\item Configurar el tiempo de envio de la información. Actualmente esta fijo en 10 segundos, factor que podria generar que se note la sobrecarga y por consiguiente que se detecte el malware.
\item Mejorar el servidor visualmente.
\item Lograr identificar el dispositivo. Ya sea por un login o por la MAC del dispositivo.
\end{itemize}