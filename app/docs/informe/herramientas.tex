\section{Herramientas utilizadas}

Para trabajar con Android vamos a necesitar el SDK, un emulador para celulares, y un entorno de programación. Todo esto viene incluido en un paquete fácil de utilizar.

Comenzamos entonces descargando el ADT Bundle (http://developer.android.com/sdk/index.html) que incluye los componentes esenciales del SKD de Android y una versión del Eclipse que incorpora ADT (Android Development Tools) para agilizar el desarrollo de aplicaciones Android. La herramienta incluye el emulador que utilizamos para simular que estamos trabajando sobre un celular y poder hacer las verificaciones más rápidamente. Para correrlo hay que seleccionar un dispositivo, nosotros utilizamos el Nexus One que es un celular básico.

Por otro lado para el servicio no hace falta utilizar algo tan pesado como otro Eclipse, así que preferimos utilizar una aplicación en PHP. Para ello necesitamos un servidor.

El WAMP (Windows Apache MySql PHP) monta un servidor completo en Windows de manera muy sencilla, simplemente instalándolo y corriéndolo

Ahora sí, podemos levantar el servidor y tener la aplicación funcionando. Para esto solo necesitamos copiar la carpeta del servidor en la carpeta www que se encuentra en el directorio en donde se instaló el WAMP
