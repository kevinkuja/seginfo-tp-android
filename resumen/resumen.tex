\documentclass[10pt,a4paper]{article}
\usepackage[utf8]{inputenc} % para poder usar tildes en archivos UTF-8
\usepackage[spanish]{babel} % para que comandos como \today den el resultado en castellano
\usepackage{a4wide} % márgenes un poco más anchos que lo usual
\usepackage[conEntregas]{caratula2}
\usepackage{float}
\usepackage[pdftex]{graphicx}
\usepackage{caption}
\usepackage{subcaption}
%Esto de abajo es para encabezado y pie de pagina
\usepackage{lastpage}
\usepackage{fancyhdr}
\usepackage{ulem}
% Simbolos matemáticos
%\usepackage{amsmath}
%\usepackage{amsfonts}
%\usepackage{amssymb}
%\usepackage{algorithm}
%\usepackage{algpseudocode}
% Descoración y gráficos
%\usepackage{fancyhdr}
%\usepackage{multirow}
%\usepackage{alltt}
\usepackage{listings}
\usepackage{color}

\definecolor{mygreen}{rgb}{0,0.6,0}
\definecolor{mygray}{rgb}{0.5,0.5,0.5}
\definecolor{mymauve}{rgb}{0.58,0,0.82}

\lstset{ %
  backgroundcolor=\color{white},   % choose the background color; you must add \usepackage{color} or \usepackage{xcolor}
  basicstyle=\footnotesize,        % the size of the fonts that are used for the code
  breakatwhitespace=false,         % sets if automatic breaks should only happen at whitespace
  breaklines=true,                 % sets automatic line breaking
  frame=single,                    % adds a frame around the code
  keepspaces=true,                 % keeps spaces in text, useful for keeping indentation of code (possibly needs columns=flexible)
  keywordstyle=\color{blue},       % keyword style
  language=SQL,                 % the language of the code
  numbers=left,                    % where to put the line-numbers; possible values are (none, left, right)
  numbersep=5pt,                   % how far the line-numbers are from the code
  numberstyle=\tiny\color{mygray}, % the style that is used for the line-numbers
  rulecolor=\color{black},         % if not set, the frame-color may be changed on line-breaks within not-black text (e.g. comments (green here))
  showspaces=false,                % show spaces everywhere adding particular underscores; it overrides 'showstringspaces'
  showstringspaces=false,          % underline spaces within strings only
  showtabs=false,                  % show tabs within strings adding particular underscores
  stepnumber=1,                    % the step between two line-numbers. If it's 1, each line will be numbered
  stringstyle=\color{mymauve},     % string literal style
  tabsize=4,                       % sets default tabsize to 2 spaces
  title=\lstname                   % show the filename of files included with \lstinputlisting; also try caption instead of title
}

\pagestyle{fancy}

\cfoot{\thepage /\pageref{LastPage} }



\newcommand\BlockIf[1]{\KwSty{If} \\ #1 \\ \KwSty{End If}}
\newcommand\BlockElseIf[1]{\KwSty{Else If} \\ #1 \\ \KwSty{End Else If}}
\newcommand\BlockElse[1]{\KwSty{Else} \\ #1 \\ \KwSty{End Else}}

\begin{document}

\titulo{Trabajo Práctico}
\subtitulo{Malware de Android}

\fecha{\today}

\materia{Seguridad Informática}
\submateria{1er Cuatrimestre de 2015}


\integrante{Castro, Alan}{}{alancastro90@gmail.com}
\integrante{Kujawski, Kevin}{459/10}{kevinkuja@gmail.com}
\integrante{Ortíz de Zárate, Juan Manuel}{}{jmanuoz@gmail.com}
\integrante{Vanecek, Juan}{}{juann.vanecek@gmail.com}



\maketitle

\newpage
%\tableofcontents
%compilar varias veces si no se actualiza el indice o el pie de pagina

%\newpage
\section{Introducción}
En este informe se detallan las caracter\'isticas de una aplicaci\'on de Android destinada al robo de informaci\'on del usuario que la utiliza.
Este tipo de aplicaci\'on es denominada Malware ya que tiene como objetivo infiltrarse o dañar un sistema sin consentimiento del propietario.
Como se detallar\'a m\'as adelante esta obtendr\'a, sin conocimiento del usuario, informaci\'on  de la ubicaci\'on del dispositivo y de los contactos contenidos en el mismo.
Toda la informaci\'on recaudada, ser\'a utilizada para armar un historial que puede tener distintos tipos de usos.
Como es de suponerse para no levantar sospechas, la aplicaci\'on se presentar\'a como un juego para dispositivos Android que a simple vista de un usuario com\'un, no informa ni visualiza ning\'un tipo de irregularidad.

\section{Actividades realizadas}


\section{Herramientas utilizadas}

Para trabajar con Android vamos a necesitar el SDK, un emulador para celulares, y un entorno de programación. Todo esto viene incluido en un paquete fácil de utilizar.
Comenzamos entonces descargando el ADT Bundle (http://developer.android.com/sdk/index.html) que incluye los componentes esenciales del SKD de Android y una versión del Eclipse que incorpora ADT (Android Development Tools) para agilizar el desarrollo de aplicaciones Android. La herramienta incluye el emulador que utilizamos para simular que estamos trabajando sobre un celular y poder hacer las verificaciones más rápidamente. Para correrlo hay que seleccionar un dispositivo, nosotros utilizamos el Nexus One que es un celular básico.


Por otro lado para el servicio no hace falta utilizar algo tan pesado como otro Eclipse, así que preferimos utilizar una aplicación en PHP. Para ello necesitamos un servidor.
El WAMP (Windows Apache MySql PHP) monta un servidor completo en Windows de manera muy sencilla, simplemente instalándolo y corriéndolo
Ahora sí, podemos levantar el servidor y tener la aplicación funcionando. Para esto solo necesitamos copiar la carpeta del servidor en la carpeta www que se encuentra en el directorio en donde se instaló el WAMP

\section{Librerías}


\section{Bibliograf\'ia}

\begin{enumerate}[nolistsep]
 \item Xuxian Jiang and Yajin Zhou (2013), Android Malware, ISBN 978-1-4614-7393-0. 
 \item StackOverflow
 \item Google
\end{enumerate}





\end{document}

